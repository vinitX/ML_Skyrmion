\documentclass[reprint,amsmath,amssymb,aps,showpacs,twocolumn,superscriptaddress,prb]{revtex4-1}

\usepackage{graphicx}
\usepackage{bm}
\usepackage{epsfig}
\usepackage{amssymb}
\usepackage{amsfonts}
\usepackage{braket}
\usepackage{color}
\usepackage{epstopdf}
\epstopdfsetup{update}
\usepackage{hyperref}
\usepackage{float}
\restylefloat{table}
\usepackage{bibentry}
\usepackage{multirow}
\usepackage[caption=false]{subfig}
\newcommand{\ba}{\begin{eqnarray}}
\newcommand{\ea}{\end{eqnarray}}
\newcommand{\bd}{\begin{displaymath}}
\renewcommand{\v}[1]{{\bf #1}}
\newcommand{\nn}{\nonumber \\}


\begin{document}
\title{Supplementary Materials for ``Machine Learning Application to Two-Dimensional Dzyaloshinskii-Moriya Ferromagnets''}

\author{Vinit Kumar Singh}
\email[Electronic address:$~~$]{vinitsingh911@gmail.com}
\affiliation{Department of Physics, Indian Institute of Technology, Kharagpur 732102, India}
\author{Jung Hoon Han}
\email[Electronic address:$~~$]{hanjemme@gmail.com}
\affiliation{Department of Physics, Sungkyunkwan University, Suwon 16419, Korea}
\date{\today}
\begin{abstract}
We detail the results of error analyses for various machine-learning predictions mentioned in the main text. 
\end{abstract}
\maketitle


\begin{table}[htb]
\begin{tabular}{ | ccc || ccc  cc  cc  cc |}
\hline
 &  XYZ-type & & & $\Delta\chi$ &  & $\Delta m$ &  & $\Delta B$ & & $\Delta T$ & \\ \hline
 &  $H_{\rm HDMZ}$ & & & 5.82 &  & 3.79 &  & 4.91 & & 5.32 & \\ \hline
 & $H_{\rm HDMZ}+ H_1$  &  & & 5.83 & & 3.85 & & 5.49 & &  5.62 & \\ \hline
 & $H_{\rm HDMZ}+ H_2$ & & & 6.01 & & 3.77 & & 7.22 & & 6.75 & \\ \hline
 & $H_{\rm HDMZ} ~ (b=2)$   & & & 7.05 & & 4.15 & & 10.1 & & 4.66 & \\ \hline
 & $H_{\rm HDMZ} ~ (b=3)$ & & & 6.46 & & 3.69 & & 11.7 & & 4.93 & \\ \hline 
 & $H_{\rm HDMZ} ~ (b=4)$ & & & 6.61 & & 4.07 & & 12.2 & & 5.89 & \\ \hline \hline
 &  XY-type & & & $\Delta\chi$ &  & $\Delta m$ &  & $\Delta B$ & & $\Delta T$ & \\ \hline
 &  $H_{\rm HDMZ}$ & & & 7.15 &  & 5.4 &  & 7.28 & & 5.23 & \\ \hline
 & $H_{\rm HDMZ}+ H_1$  &  & & 7.52 & & 6.2 & & 8.5 & &  5.42 & \\ \hline
 & $H_{\rm HDMZ}+ H_2$ & & & 8.25 & & 7.76 & & 11.8 & & 6.37 & \\ \hline \hline
 &  Z-type & & & $\Delta\chi$ &  & $\Delta m$ &  & $\Delta B$ & & $\Delta T$ & \\ \hline
 &  $H_{\rm HDMZ}$ & & & 5.98 &  & 3.28 &  & 5.14 & & 6.33 & \\ \hline
 & $H_{\rm HDMZ}+ H_1$  &  & & 6.09 & & 3.2 & & 5.56 & &  6.48 & \\ \hline
 & $H_{\rm HDMZ}+ H_2$ & & & 5.65 & & 3 & & 7.2 & & 6.66 & \\ \hline
\end{tabular}\label{table:1}
\caption{Averaged variance between predicted and actual values of $(\chi, m, B, T)$.}
\end{table}

Listed in Table 1 are the errors in the machine-predicted values of $(\chi, m, B, T)$. The error estimation is done by the formula 
%
\ba \Delta X = \sqrt{ { \sum_i (X_{{\rm predicted}, i} - X_{{\rm actual}, i} )^2 \over N} }. \ea
%
Here $X=\chi, m, B, T$ and $1\le i \le N$ ranges over all the test configurations. Input data types are classified as $xyz$, $xy$, and $z$, according to all three components, only $xy$-component, and only $z$-component of the local magnetization vector $\v n_i$ being used for training and testing. The pure case $H_{\rm HDMZ}$ refers to the choice $D/J=\sqrt{6}$ corresponding to the spiral period $\lambda=6$. The two disordered Hamiltonians we considered in the main text are shown in the rows with $H_{\rm HDMZ}+ H_1$ and $H_{\rm HDMZ}+ H_2$. The sample size is $N= 20\times 20 \times 100$. 

For $b=2,3,4$, only the pure Hamiltonian $H_{\rm HDMZ}$ was used with $D/J$ values corresponding to $\lambda=12,18,24$, respectively. The resulting raw data is compressed according to the block-spin rule (mentioned in the text) before being subject to machine prediction. The predicted values of $\chi, m, b, T$ are then compared to $\chi', m', B', T'$, which is related to the raw value through the scaling relation $\chi'/\chi = b^\#$. The exponents used are 0, 0, 2.32, and 0.73, respectively. For example, the variance $\Delta B$ in the case of $b=2$ is obtained from
%
\ba \Delta B = \sqrt{ { \sum_i (B_{{\rm predicted}, i} - B_{{\rm actual}, i} 2^{2.32} )^2 \over N} } \ea
%
where $B_{{\rm actual}, i}$ is the magnetic field used in the generation of the $\lambda=12$ Monte Carlo configuration. The sample size was $N=14\times 11 ~ (b=2)$, $N=14\times 9 ~ (b=3)$, and $N=15\times 7 ~ (b=4)$. 

%\begin{table}[htb]
%\begin{tabular}{ | ccc || ccc  cc  cc  cc |}
%\hline
% &  XY-type & & & $\Delta\chi$ &  & $\Delta m$ &  & $\Delta B$ & & $\Delta T$ & \\ \hline
% &  $H_{\rm HDMZ}$ & & & 7.15 &  & 5.4 &  & 7.28 & & 5.23 & \\ \hline
% & $H_{\rm HDMZ}+ H_1$  &  & & 7.52 & & 6.2 & & 8.5 & &  5.42 & \\ \hline
% & $H_{\rm HDMZ}+ H_2$ & & & 8.25 & & 7.76 & & 11.8 & & 6.37 & \\ \hline
%\end{tabular}\label{table:2}
%\caption{Averaged variance between predicted and actual values of $(\chi, m, B, T)$  for $xy$-type training/testing data.}
%\end{table}


%\begin{table}[htb]
%\begin{tabular}{ | ccc || ccc  cc  cc  cc |}
%\hline
% &  Z-type & & & $\Delta\chi$ &  & $\Delta m$ &  & $\Delta B$ & & $\Delta T$ & \\ \hline
% &  $H_{\rm HDMZ}$ & & & 5.98 &  & 3.28 &  & 5.14 & & 6.33 & \\ \hline
% & $H_{\rm HDMZ}+ H_1$  &  & & 6.09 & & 3.2 & & 5.56 & &  6.48 & \\ \hline
% & $H_{\rm HDMZ}+ H_2$ & & & 5.65 & & 3 & & 7.2 & & 6.66 & \\ \hline
%\end{tabular}\label{table:3}
%\caption{Averaged variance between predicted and actual values of $(\chi, m, B, T)$  for $z$-type training/testing data.}
%\end{table}



\begin{thebibliography}{99}
\end{thebibliography}

\end{document}
